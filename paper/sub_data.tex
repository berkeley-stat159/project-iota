\subsection{Bonferroni Correction}

The Bonferroni correction is a multiple-comparison correction used when several
dependent or independent statistical tests are being performed simultaneously.
In order to avoid a lot of spurious positives, the alpha value needs to be 
lowered to account for the number of comparisons being performed.

Suppose that the model is true so that $t_{1}, \ldots, t_{n}$ have the $t(n-p-2)$
distribution. They are not necessarily independent, and we would expect about
$n \alpha$ number of $t_{1}, \ldots, t_{n}$ to be larger in absolute value than 
than $\alpha/2$ critical value of the $t(n-p-2)$. Therefore, we would be tagging 
$n \alpha$ of the subjects as outliers even when there are no outliers in the 
data.

To counter this, one takes a value of $\alpha$ much smaller than 0.05 while 
testing whether the $i$th observation is an outlier or not. A particularly conservative 
value is $\alpha = 0.05/n$. In this case, one can show that the probability that at 
least one subject is tagged as an outlier when in fact there are non is almost 0.05. 
