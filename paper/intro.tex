\section{Introduction}

\par The paper of which our research is based is titled ``Working memory related
brain network connectivity in individuals with schizophrenia and their
siblings'' \cite{repovs_barch1, repovs_barch2}. Schizophrenia is a chronic,
severe, and disabling brain disorder. Previous studies have shown that changes
in the function of a single brain region, or even a brain system, cannot explain
the functional impairments seen in this illness \cite{kircher_thienel, repovs_barch1}.
However, Repovs and Barch attempts to show that individuals with schizophrenia
have reduced connectivity within and between neural networks, which could be a
forward step to understanding schizophrenia. 

Repov and Barch systematically examined changes in functional connectivity across
rest and different task states in order to make inferences on characteristics
of schizophrenia. They assessed connectivity using blood oxygen level dependent
(BOLD) time series acquired using fMRI. They found four types of participates, 
individuals with schizophrenia, the siblings of individuals with schizophrenia, healthy 
controls and the siblings of healthy controls. They designed three working memory 
loads of an N-back task and designated four regions of interest (ROIs). The four brain
networks are: (1) Dorsal fronto-parietal network (FP), (2) Cingulo-opercular
network (CO), (3) Cerebellar network (CER) and (4) "Default mode" network (DMN).
The objective of the study is to examine the altered functional connectivity
within and between these four brain networks when the participant perform a
designated N-back task. With ANOVA analysis, it was found that individuals with
schizophrenia and their siblings showed consistent reductions in connectivity
between both the FP and CO networks with the CER network. 

We narrowed our focus of analysis on a single schizophrenic
subject. The fMRI data in the study shows the changes in
blood flows in the brain. Analyzing this data can help us understand how
schizophrenic brains respond during task performances. We detailed in the following the 
steps we took to analyze the data set. We attempted both spatial and temporal 
smoothing on the voxels in order to reduce noise variance. Having preprocessed the 
data, we fit a multiple linear regression model using two different sets of experiment
conditions. Having obtained coefficient estimates, we performed a Student t-test on each 
voxel to examine whether the particular voxel has show significant activity subject to 
appropriate thresholds. Performing t-tests relying on a set of assumptions that we need 
to validate. We performed the Shaprio-Wilk test to check the normality assumption. In 
addition, we performed a heteroskedasticity test (White test) that checked  the constant 
variance and linear model assumption. We also performed ANOVA tests that compared
different sub models. We performed cross validation (CV) to select the optimal model for 
our data. And lastly, we compared 0-back and 2-back tests through t-test. 

