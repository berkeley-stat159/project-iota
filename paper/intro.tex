\section{Introduction}

\par The paper of which our research is based is titled ``Working memory related
brain network connectivity in individuals with schizophrenia and their
siblings'' \cite{repovs_barch1, repovs_barch2}. Schizophrenia is a chronic,
severe, and disabling brain disorder. Previous studies have shown that changes
in the function of a single brain region, or even a brain system, cannot explain
the functional impairments seen in this illness \cite{kircher_thienel, repovs_barch1}.
However, Repovs and Barch attempts to show that individuals with schizophrenia
have reduced connectivity within and between neural networks, which could be a
forward step to understanding schizophrenia. 

Repov and Barch systematically examine changes in functional connectivity across
rest and different task states in order to make an inference on characteristics
of schizophrenia. They asses connectivity using blood oxygen level dependent
(BOLD) time series acquired using fMRI. They find four types of participates, individuals with
schizophrenia, the siblings of individuals with schizophrenia, healthy controls
and the siblings of healthy controls. They designed three working memory loads
of an N-back task and designated four regions of interest (ROIs). The four brain
networks are: (1) Dorsal fronto-parietal network (FP), (2) Cingulo-opercular
network (CO), (3) Cerebellar network (CER) and (4) "Default mode" network (DMN).
The objective of the study is to examine the altered functional connectivity
within and between these four brain networks when the participant perform a
designated N-back task. With ANOVA analysis, it is found that individuals with
schizophrenia and their siblings show consistent reductions in connectivity
between both the FP and CO networks with the CER network. 

We narrow our focus and analysis on a single schizophrenic
subject. The fMRI data in the study shows the changes in
blood flow in the brain, and analyzing this data can help us understand how
schizophrenic brains respond during task performance. We detail in the following the steps
we take to analyze the data. We attempted both spatial and temporal smoothing on
the voxels in order to reduce noise. We also performed spectral analysis on fMRI data to
correct noise in the data. After preprocessing the data, we fit
multiple linear regression model using two different sets of experiment
conditions. Having obtained coefficient estimates, we performed Student t-tests
on each voxel to examine whether the particular voxel has show significant
activity subject to appropriate thresholds. Performing t-tests relying on a set
of assumptions that we need to validate. We performed the Shaprio-Wilk test to
check the normality assumption.

