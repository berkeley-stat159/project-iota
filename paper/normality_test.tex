In our analysis part, we use linear model to fit real data. There are several
assumptions on linear modeling, one of which is normality assumption on errors 
term. Normality of the errors can be checked by checking normality of the 
residuals. The usual way is plotting a Q-Q plot and check whether it forms a 
straight line. However in this paper, since there are around 200,000 voxels
it would be hard to plot all of them and check for normality. Thus we use 
Shapiro-Wilk test in the paper to test for normality.

In Shapiro-Wilk test, a small p-value indicates non-linearity. We write up a
function to compute p-value of each voxel and plot those p-values in three
brain perspectives. There are three linear models used in the paper, the below
6 plots are p-value plots, first three of them are p-values for three different
linear models using raw data. The other three plots are p-values for three 
different linear models using smoothed data. 









